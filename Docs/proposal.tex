\documentclass[10pt,a4paper,twoside,noindent]{report}
\usepackage[utf8]{inputenc}
\usepackage{amsmath}
\usepackage{amsfonts}
\usepackage{amssymb}
\author{Jason Forte}
\title{Final Year Project Proposal}

%% Set noindent for the entire document
\setlength{\parindent}{0px}


%% Set parskip for spaces between paragraphs
\setlength{\parskip}{0.25cm plus3mm minus2mm}

\begin{document}

\thispagestyle{empty}
\section*{Project Proposal - Biometric Vein Scanner utilising near infra-red absorption}
\subsection*{Introduction}

There has been extensive developments in the field of biometrics. The field is mainly dominated by big players such as finger-print and facial feature recognition. Over the past few years there has been a vast number of aspects that are gaining momentum. One of these is the use of subcutaneous veins in the hand as a means of identification.

The benefit of such a system would be in the relative difficulty in accessing vein patterns. Because they are under the skin, a system employing this method of verification would be hard to fool.

Acquisition of such patterns involves the use of near-infra-red light. This light (typically of wavelengths between 880 nm and 940 nm) has the property that allows it to be absorbed by the blood contained in veins. By imaging the lit area with a camera sensitive to infra-red radiation the reflection will show a distinct pattern that can then be processed using digital signal processing methods.

\subsection*{Background and Scope}

The scope of this project is to develop the hardware to acquire dorsal (back of hand) vein images and then to process the results with the aim of extracting the biometric information.

Along with constructing the apparatus to acquire images an API will be generated to allow accessing the device and intermediate processing capabilities.

The project will not, however, cover the classification procedure which allows the matching of a pattern to a reference pattern already stored in the database.

\subsection*{Ethics with Biometrics}

While the classification is not part of the project it is important that care be taken in handling any experimental data that is obtained.

Although in it's infancy, the dorsal vein image processing approach is experiencing a more rapid development at this time. There could be future prospects for extensive use of the technology in applications such as banking, healthcare etc.

It is important to ensure that any data captured by the device while in the process of testing be disassociated from the identification of the participant. In order to allow this, a random number will be assigned to any image captured by the device. The number will be the only point of reference for the image or derivatives of the image from that point on.

Names and identification will not be captured or recorded but skin colour may be recorded if it turns out that the scan system gives varying results dependent on skin colour. There is doubt, however, that there will be any effect on the scans due to skin colour.

\end{document}